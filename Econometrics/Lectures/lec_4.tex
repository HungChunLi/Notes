\chapter{Instrumental Variable}

\lecture{4}{13 Oct. 08:00}{First Lecture}
\lecture{5}{13 Oct. 08:00}{First Lecture}

\section{Instrumental Variable Approach}

We now see some common environment you'll need to complete your note.

\subsection*{Wald Estimator}
The model \( y_i = \alpha + \beta x_i + u_i \) has one endogenous variable and no covariates.
The instrument \( z_i \) is a binary variable consistting of 0 and 1.
\begin{align*}
    \mathrm{E}[y_i | z_i ] &= \mathrm{E}[ \alpha + \beta x_i + u_i | z_i ] \\
    &= \mathrm{E}[ \alpha + \beta x_i + u_i | z_i ] \\
    &= \alpha + \beta \mathrm{E}[ x_i | z_i ] + \underbrace{\mathrm{E}[ u_i | z_i ]}_{0} \\ 
    \Rightarrow \mathrm{E}[y_i | z_i = 1] - \mathrm{E}[y_i | z_i = 0] &= \beta \left( \mathrm{E}[ x_i | z_i = 1] - \mathrm{E}[ x_i | z_i = 0] \right) \\
    \Rightarrow \beta &= \frac{\mathrm{E}[y_i | z_i = 1] - \mathrm{E}[y_i | z_i = 0]}{\mathrm{E}[ x_i | z_i = 1] - \mathrm{E}[ x_i | z_i = 0]}.
\end{align*}
This is called the \textbf{Wald Estimator}.

\subsection*{Consistency of IV Estimator}
IV estimator is \text{biased} and \text{inefficient}, we use it because it is \text{consistent}.
\begin{align*}
    \textrm{plim} \ \hat{\beta}_{\text{IV}} &= \mathrm{plim} \ \frac{\mathrm{Cov}(z, y)}{\mathrm{Cov}(z, x)} \\
    &= \frac{\mathrm{Cov}(z, \alpha + \beta x + u)}{\mathrm{Cov}(z, x)} \\
    &= \frac{\mathrm{Cov}(z, \alpha)}{\mathrm{Cov}(z, x)} + \frac{\mathrm{Cov}(z, \beta x)}{\mathrm{Cov}(z, x)}+ \frac{\mathrm{Cov}(z, u)}{\mathrm{Cov}(z, x)} \\
    &= \beta + \frac{\mathrm{Cov}(z, u)}{\mathrm{Cov}(z, x)} \\
    &= \beta + \frac{\mathrm{Corr}(z, u) \sigma_z \sigma_u}{\mathrm{Corr}(z, x) \sigma_z \sigma_x} \\
    &= \beta + \underbrace{\frac{\mathrm{Corr}(z, u)}{\mathrm{Corr}(z, x)} \times \frac{\sigma_u}{\sigma_x}}_{\text{Bias}} \\
\end{align*}
The equation shows that if \( \mathrm{Corr}(z, u) = 0 \) (exogenous) and \( \mathrm{Corr}(z, x) \neq 0 \) (relevance), then \( \mathrm{plim} \ \hat{\beta}_{\text{IV}} = \beta \), the IV estimator exists and is consistent.
When \( \mathrm{Corr}(z, x) \) is small, the bias can be large, we call this the \textbf{weak instrument problem}. 

\subsection*{Matrix Representation}
Later


\section{Two-Stage Least Squares (2SLS)}

\begin{itemize}
    \item \text{Just-identified}: the number of instruments equals the number of endogenous regressors.
    \item \text{Over-identified}: the number of instruments exceeds the number of endogenous
\end{itemize}

\( x_1 \) is endogenous variables and \( z_1, z_2, z_3 \) is exogenous variables.
As \( z_1, z_2, z_3 \) are all uncorrelated with \( u \), any linear combination of them is also uncorrelated with \( u \).
\begin{align*}
    y &= \alpha + \beta_1 \hat{x}_1 + \beta_2 z_1 + u \\
    x_1 &= \pi_0 + \pi_1 z_1 + \pi_2 z_2 + \pi_3 z_3 + v
\end{align*}

\subsection*{Matrix Representation}
\[
    y = X \beta + u, 
\] 
where \( X = [x_1, \dots, x_k ] \) contains both endogenous and exogenous variables. 
\( Z = [z_1, \dots, z_m] \) contains all exogenous variables with larger dimension than \( X \) (i.e., \( m \geq k \)), some columns of \( Z \) may be used as instruments for the endogenous variables in \( X \).


The columns of \( X \) are projected onto the space spanned by the columns of \( Z \):
\[
    \hat{X} = Z (Z'Z)^{-1} Z' X = P_Z X,
\]
where \( P_Z \) is the projection matrix onto the column space of \( Z \).